\documentclass{article}
\usepackage{example_sentences}
\begin{document}
%
\noindent Examples are typeset like this:
    \begin{examples}
        \item This is an example.
        \item This is a second example.
              \begin{examples}
                  \item It has sub-examples.
                  \item<*> Ungrammatical are this sentence.
                        \begin{examples}
                            \item And yet, it has sub-examples. Like this one.
                            \item And this one.
                        \end{examples}
              \end{examples}
    \end{examples}
There are some advanced features, like concise label assignment:
\begin{examples}
    \item(foo) This example will have the label \verb|foo|.
    \item(bar)<\#> This example will have the label \verb|bar|, and will be marked 
                unacceptable.
\end{examples}
%
Now we can refer to the above examples by using \verb|\ex{foo}| and \verb|\exref{bar}|, like this: \ex{foo} and \exref{bar}.

Crossreferences, are also supported. if you want to repeat an example with its original 
number, use \verb|\item{label}|:
\begin{examples}
    \item{foo} This example will have the label `foo'.
\end{examples}
%
And, as usual, if you want to manually supply the example number, use \verb|\item[(mynumber)]|:
\begin{examples}
    \item[(42)] This example is numbered 42.
\end{examples}
%
Finally, if you have you use a single example, you can use the singular form of the environment name:
\begin{example}
    \item This is a lonely example.
\end{example}
\end{document}