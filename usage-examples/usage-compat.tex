\documentclass{article}
%
\usepackage[compat]{example-sentences}
%
\begin{document}
%
\noindent In \verb|[compat]| mode, example lists behave exactly
like \verb|enumerate| lists. This means some things are more cumbersome,
for example the diacritic in (\ref{ungrammatical}) below.
    \begin{examples}
        \item This is an example.
        \item This is a second example.
              \begin{examples}
                  \item It has sub-examples.
                  \item \diacritic{*} Ungrammatical are this sentence.\label{ungrammatical}
                        \begin{examples}
                            \item And yet, it has sub-examples. Like this one.
                            \item And this one.
                        \end{examples}
              \end{examples}
    \end{examples}
%
And the reference convenience commands will be turned off, so we have to use: 
(\ref{ungrammatical}). Or use a fancy reference package like \verb|cleverref.sty| (recommended).

For crossreferences, we have to write \verb|\item[(\ref{label})]| (which is not too bad).
\begin{examples}
    \item[(\ref{ungrammatical})] \diacritic{*} Ungrammatical are this sentence.
\end{examples}
%
\textbf{Note:} All of the above are perfectly valid ways of doing things in normal mode,
so even if your collaborators insist on fancy convenience commands, you can do things
the old school way in your part of the document, if you are so inclined.
\end{document}