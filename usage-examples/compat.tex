\documentclass{article}
%
\usepackage[compat]{example_sentences}
%
\begin{document}
%
\noindent In \verb|[compat]| mode, example lists behave exactly
like \verb|enumerate| lists. This means some things are more cumbersome,
for example the diacritic in (\ref{ungrammatical}) below.
    \begin{examples}
        \item This is an example.
        \item This is a second example.
              \begin{examples}
                  \item It has sub-examples.
                  \item \diacritic{*} Ungrammatical are this sentence.\label{ungrammatical}
                        \begin{examples}
                            \item And yet, it has sub-examples. Like this one.
                            \item And this one.
                        \end{examples}
              \end{examples}
    \end{examples}
Consise label assignment will not work, so you'll have to use \verb|\label|:
\begin{examples}
    \item\label{foo} This example will have the label \verb|foo|.
    \item\label{bar}\diacritic{\#} This example will have the label \verb|bar|, and will be marked 
                unacceptable.
\end{examples}
%
And the reference convenience commands will be turned off, so we have to use: 
(\ref{foo}) and (\ref{bar}). Or use a fancy reference package like \verb|cleverref.sty| (recommended).

For crossreferences, we have to write \verb|\item[(\ref{label})]| (which is not too bad).
\begin{examples}
    \item[(\ref{foo})] This example will have the label `foo'.
\end{examples}
%
Single example, however, can still be typeset with \verb|\begin{example}\end{example}|:
\begin{example}
    \item This is a lonely example.
\end{example}

\textbf{Note:} All of the above are perfectly valid ways of doing things in normal mode,
so even if your collaborators insist on fancy convenience commands, you can do things
the old school way in your part of the document, if you are so inclined.
\end{document}