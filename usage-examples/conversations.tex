\documentclass{article}
%
\usepackage[conversations]{example_sentences}
%
\begin{document}
%
\noindent From version 0.6.0 on, \verb|example_sentences.sty| has a \verb|[conversations]| 
option, which makes available the \verb|conversation| environment for typesetting dialogues.

It is used like this:
\begin{examples}
    \item(godot) [A country road. A tree.]
    \begin{conversation}
        \item[Estragon:] Nothing to be done.
        \item[Vladimir:] I am beginning to come round to that opinion. All my life I've tried to put it from me, saying Vladimir, be reasonable, you haven't yet tried everything. And I resumed the struggle.

        So you are here again. 
        \item[Estragon:] Am I? 
    \end{conversation}
\end{examples}

Standardly, the longest label determines how much the text will be indented. But 
if you have some exceptionally long items, it will often look better to set a 
fixed label length, and have a newline after the label if it exceeds
that length. Luckily, \verb|enumitem| provides just such a style. So you can 
simply do this:
\begin{examples}
    \item \begin{conversation}[leftmargin=2cm,style=nextline]
        \item[Question:] What would you think it is worth telling [future 
                         generations] about the life you've lived, and the 
                         lessons you've learned from it?
                         
        \item[Bertrand, Third Earl Russel:] 
              I should like to say two things, one illectual, and one moral.

              The intellectual thing I should like to say to them is this: When you
              are studying any matter, or considering any philosophy, ask yourself only:
              What are the facts, and what is the truth that the facts bear out. [...]

              The moral thing I should wish to say to them is very simple. I shoud say: Love is wise, hatred is foolish. In this world, which is getting more and
              more closely interconnected, we have to learn to tolerate each other. We
              have to learn to put up with the fact that some people say things we don't
              like. 

              We can only live together in that way. And if we are to live together and 
              not die together, we must learn a kind of charity and a kind of tolerance. 
              Which is absolutely vital to the continuation of human life on this planet.
                  \end{conversation}
\end{examples}
If you use the standard mode (where the width of the labels is calculated), and 
you need a little extra space (to typeset a diacritic, say),
you can manipulate the length \verb|\conversationindent|, as in:
\begin{examples}
  \setlength{\conversationindent}{1em} % Default is 0.5em
  \item \begin{conversation}
        \item[Estragon:]<*> Be to nothing done.  
    \end{conversation}  
\end{examples}
\end{document}