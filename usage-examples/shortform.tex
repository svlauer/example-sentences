\documentclass{article}
%
\usepackage[shortform]{../example_sentences}
%
\begin{document}
%
\noindent With the \verb|[shortform]| option, examples can be typeset like this:
    \begin{exe}
        \ex This is an example.
        \ex This is a second example.
              \begin{exe}
                  \ex It has sub-examples.
                  \ex<*> Ungrammatical are this sentence.
                        \begin{exe}
                            \ex And yet, it has sub-examples. Like this one.
                            \ex And this one.
                        \end{exe}
              \end{exe}
    \end{exe}
Of course, examples can be in footnotes.%
\footnote{%
  Examples in footnotes are numbered differently, but still can be referred
  to via \ex{footnote-example}:
  \begin{examples}
      \item(footnote-example) This is an example in a footnote.
  \end{examples}
}\textsuperscript{,}\footnote{%
  \begin{examples}
      \item The footnote counter starts anew in each footnote.
  \end{examples}
}

There are some advanced features, like concise label assignment:%

\begin{exe}
    \ex(foo) This example will have the label \verb|foo|.
    \ex(bar)<\#> This example will have the label \verb|bar|, and will be marked 
                unacceptable.
\end{exe}
%
Now we can refer to the above examples by using \verb|\ex{foo}| and \verb|\exref{bar}|, like this: \ex{foo} and \exref{bar}.

And if you want to manually supply the example number, use \verb|\ex[(mynumber)]| (as with \verb|\item|):
\begin{exe}
    \ex[(42)] This example is numbered 42.
\end{exe}
%
This can be used to do cross-references, like this:%
\begin{exe}
    \ex[\exref{foo}] This example will have the label `foo'.
\end{exe}

In \verb|[shortform]| mode, all the usual commands remain available, 
and you can mix the two freely (though maybe you should not do that):
\begin{exe}
  \item<*> A sentence.
  \ex Another one.
\end{exe}
\begin{examples}
  \ex<??> Another sentence.
\end{examples}

\end{document}